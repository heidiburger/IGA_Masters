\documentclass[a4paper]{report}
\usepackage{a4wide}
\usepackage{verbatim}
\usepackage{amsmath}
\usepackage{amssymb}
\usepackage{float}
\usepackage{graphicx}
\usepackage{caption}
\usepackage{subcaption}
\author{Heidi Burger\BRGHEI006}
\title{Isogeometric Analysis}
\begin{document}
	\newpage
	\begin{figure}[h]
	\centering
	\includegraphics[scale=0.7]{uct_logo}
	\end{figure}
	\begin{center}
		\textbf{
			\begin{large}
				Masters Dissertation\\
				.....\\
				Isogeometric Analysis\\
				.....\\
				BRGHEI006\\
				.....\\
				Heidi Burger\\
				.....\\
				\today
		\end{large}}
	\end{center}


\newpage \normalsize
\chapter{NURBS}
NURBS stand for Non-uniform rational B-splines and is a set of mathematical functions that describe a curve (one dimension), surface (two dimensions), volume (three dimensions). NURBS are a superset of B-splines and can be described as rational functions of B-splines. NURBS are used over B-splines, as they give better representations of conical, spherical and ellipsoidal shapes, among others. In this report, B-splines will be described first as NURBS are just a small change from B-splines, which are easier to understand. \\

\noindent B-splines are described parametrically. For 1D, there is only one parameter: Xi. This  parameter set is called a knot vector, which is always represented in ascending order. The knot spans are the differences between each knot pair in the knot vector. If the knots are evenly spaced, the knot vector is referred to as an uniform knot vector. Conversely, non-uniform knot vectors are unevenly spaced. 

Eg. $Xi=\{0,0,0,1,2,3,4,5,5,5\}$\\

\noindent If a knot occurs more than once, adjacently, it is said to have a certain multiplicity

\chapter{Definitions}


\end{document}