\documentclass[a4paper]{report}
\usepackage{a4wide}
\usepackage{verbatim}
\usepackage{amsmath}
\usepackage{amssymb}
\usepackage{float}
\usepackage{graphicx}
\usepackage{caption}
\usepackage{subcaption}
\author{Heidi Burger\BRGHEI006}
\title{Isogeometric Analysis}
\begin{document}
	\newpage
	\begin{figure}[h]
	\centering
	\includegraphics[scale=0.7]{uct_logo}
	\end{figure}
	\begin{center}
		\textbf{
			\begin{large}
				Masters Dissertation\\
				.....\\
				Isogeometric Analysis\\
				.....\\
				BRGHEI006\\
				.....\\
				Heidi Burger\\
				.....\\
				\today
		\end{large}}
	\end{center}


\newpage \normalsize
\chapter{2D scalar Manual}

\section{Weak Form}
	A two dimensional heat problem with the governing equation (strong form):
	
	\begin{equation}
		\underline{\nabla}^T \underbar{q} - S = 0 \quad \text{on $\Omega$} 
		\label{governing2Dscalar}
	\end{equation}
	where $q=$ is the temperature and $S$ is the source term defined as:
		
	\begin{eqnarray}
		\underline{q}=-\underline{k}\underline{\nabla}T \\
		S = 0
	\end{eqnarray}
	where $T$ is the temperature of the domain and $S$ is the source term and the boundary conditions are defined as:
	
	\begin{eqnarray}
		\underline{q}=\bar{\underline{q}}  \quad \text{on $\Gamma_N$} \\
		T=\bar{T}  \quad \text{on $\Gamma_N$}
	\end{eqnarray}
	
	\noindent can be solved by using Isogeometric Analysis (IGA). The above governing equation can be converted to the weak form in order to solve the equation with NURBS. \\
	
	\noindent To convert the strong form to the weak form, the governing equation is multiplied by a test function ($w$) and integrated across the domain ($\Omega$) as seen in equations below. 
	
	\begin{eqnarray}
		\int_{\Omega} (w \nabla^T) \underline{q}  d\Omega - \int_{\Omega} wS d\Omega  =0\\
		\int_{\Omega} (\nabla^T w) \underline{q}  d\Omega - \oint_{\Gamma_N} w\underline{q}^T \underline{n} d\Gamma + \int_{\Omega} wS d\Omega =0 \\
		\int_{\Omega} (\nabla^T w) \underline{D\nabla T}  d\Omega = - \oint_{\Gamma_N} w\underline{q}^T \underline{n} d\Gamma + \int_{\Omega} wS d\Omega  
		\label{2DscalarWF}
	\end{eqnarray}
	
	\noindent Find $T \in H'$ so that $T=\bar{T}$ on $\Gamma_D$ that satisfies equation \ref{2DscalarWF} where the test functions $w\in H'$ and $w=0$ on $\Gamma_D$.

\section{Bubnov-Garlerkin form}
	
	Using the Bubnov-Garlerkin method \textbf{(insert reference)}, where the same shape functions ($\Phi_I$) are used for both the field variable (temperature in this case) and the test function (w). The Bubnov-Garlerkin equation form can be seen in equation \ref{2DscalarBGF} below:
	
	\begin{equation}
		\underbrace{\int_{\Omega} \frac{d\underline{R}}{d\underline{x}}^T \underline{D} \frac{d\underline{R}}{d\underline{x}} d\Omega}_{\underline{K}} \underbrace{\underline{d}}_{\underline{d}} = \underbrace{\int_{\Omega} \underline{R}^T s d\Omega - \oint_{\Gamma_N} \underline{N}^T \bar{\underline{q}} d\Gamma }_{\underline{F}}
		\label{2DscalarBGF}
	\end{equation}
	
	\noindent where $w$ represents the nodal displacement variations, $\underline{R}$ is a vector of the products of the basis functions associated with each node on the control polygon element and $\underline{N}$ is a vector of basis functions associated with each node along the one side of the element. \textbf{(Check nomenclature with Ernesto. And define elements.)}


\section{Calculating the components of the Bubnov-Garlerkin form}

	In each parametric direction ($\xi$ and $\eta$), basis functions can be described for each element, depending on the basis function order ($p$ and $q$), knot vector ($\Xi$ and $H$) and NURBS weights ($w$).
	
	\noindent The following equations describe B-spline basis functions, depending on the order required and knot vector length. 
	\begin{equation}
		N^{Bspline}_{i,p}(\xi)=\frac{\xi - \xi_i}{\xi_{i+p}-\xi_i}N_{i,p-1}(\xi)+\frac{\xi_{i+p+1} - \xi}{\xi_{i+p+1}-\xi_{i+1}}N_{i+1,p-1}(\xi)
	\end{equation}
	
	where when p=0:
	\begin{equation}
		N_{i,p}(\xi)=\begin{cases}
		1, & \text{if $\xi_i \geq \xi > \xi_{i+1}$ } \\
		0, & \text{otherwise}.
		\end{cases}
	\end{equation}
	
	\noindent The NURBS basis function can be made of the B-splines ($N^{Bslpline}$) as can be seen in equation \ref{NURBSbasisFunction} below:
	
	\begin{equation}
		N^{NURBS}_{i,p}(\xi)=\frac{N^{Bspline}_{i,p}(\xi)w_i}{\sum_{j=1}^{n}N^{Bspline}_{j,p}(\xi)w_j}
		\label{NURBSbasisFunction}
	\end{equation}
	
	\noindent $\underline{R}$ is a vector of the products of basis functions. Each entry is a product of the basis functions associated with each direction. $N_i$ refers to $N^{NURBS}_{i,p}(\xi)$ in the $\xi$ direction and $M_j$ refers to $N^{NURBS}_{j,q}(\eta)$ in the $\eta$ direction. $i$ and $j$ refer to the basis function numbers associated with a certain element.
	
	\begin{equation}
		\underline{R}=[N_1M_1,N_2M_1,N_3M_1,N_1M_2,N_2M_2,N_3M_2,N_1M_3,N_2M_3,N_3M_3]
	\end{equation}
	
	\noindent To calculate the derivatives of $\underline{R}$ with respect to the physical domain ($x$ and $y$), the derivatives with respect to the parametric domain is to be calculated first. This can be done by take the derivatives of each component in $\underline{R}$ as illustrated below:
	
	\begin{eqnarray}
		\frac{dR}{d\xi} =\Bigg[\frac{dN_1}{d\xi}M_1, \frac{dN_2}{d\xi}M_1, \frac{dN_3}{d\xi}M_1, \frac{dN_1}{d\xi}M_2, \frac{dN_2}{d\xi}M_2, \frac{dN_3}{d\xi}M_2, \frac{dN_1}{d\xi}M_3, \frac{dN_2}{d\xi}M_3, \frac{dN_3}{d\xi}M_3 \Bigg] \\
		\frac{dR}{d\eta} =\Bigg[\frac{dM_1}{d\eta}N_1, \frac{dM_2}{d\eta}N_1, \frac{dM_3}{d\eta}N_1, \frac{dM_1}{d\eta}N_2, \frac{dM_2}{d\eta}N_2, \frac{dM_3}{d\eta}N_2, \frac{dM_1}{d\eta}N_3, \frac{dM_2}{d\eta}N_3, \frac{dM_3}{d\eta}N_3 \Bigg]
	\end{eqnarray}
	
	\noindent To convert the parametric derivative ($\frac{d\underline{R}}{d\underline{\xi}}$) to the physical derivative ($\frac{d\underline{R}}{d\underline{x}}$), the parametric Jacobian is required which can be calculated using the parametric derivatives and the control points associated with the element as shown in equation \ref{paraJaco}.
	
	\begin{equation}
		\begin{bmatrix}
		\frac{dR}{d\xi} \underline{x} & \frac{dR}{d\eta} \underline{x} \\
		\frac{dR}{d\xi} \underline{y} & \frac{dR}{d\eta} \underline{y} \\
		\end{bmatrix}
		\label{paraJaco}
	\end{equation}











\end{document}